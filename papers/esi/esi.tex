\documentclass[a4paper,12pt]{article}
%\usepackage{mathptm}
\usepackage{amsfonts}
%\usepackage{smart}
%\usepackage{float}
%\usepackage{latexsym}
%%%%%%%%%%%%%%%%%%%%%%%% Equation counting %%%%%%%%%%%%%%%%
%\link{equation}{section}\toheight1
\newcommand{\clth}{\setcounter{thm}{0}}
\newcommand{\sectionnew}[1]{\section{#1}\clth}
%%%%%%%%%%%%%%%%%%%%%%%%%%%%%%%%%%%%%%%%%%%%%%%%%%%%%%%%%%%%

\catcode`\@=11
\def\numberbysection{\@addtoreset{equation}{section}
        \def\theequation{\thesection.\arabic{equation}}}

\newtheorem{thm}{Theorem}[section]
\newcommand{\beq}{\begin{equation}}
\newcommand{\eeq}{\end{equation}}
\newcommand{\beqa}{\begin{eqnarray}}
\newcommand{\eeqa}{\end{eqnarray}}
\newcommand{\nn}{\nonumber \\}
\newtheorem{prop}[thm]{Proposition}
\newcommand {\np}[1]{ {\mathrm{:}}{#1}{\mathrm{:}} } %norm.pr
\newcommand {\sg}[1]{ {\mathrm{sgn}}({#1}) }
\newcommand {\I}[1]{ {\mathrm{Im}}\, {#1} }
\def\eop{\hbox{\vrule width 6pt height 6pt depth 0pt}}
\def \a {\underline{\alpha}}
\def \k {\kappa}
\def \l {\underline{\lambda}}
\def \L {\underline{\Lambda}}
\def \La {\underline{\Lambda}^{(\alpha)}}
\def \Lb {\underline{\Lambda}^{(\beta)} }
\def \s {\sigma}
\def \r {\rho}
\def \e {\underline{\mathrm{e}}}
\def \Q {\underline{Q}}
\def \q {\underline{\mathrm{q}}}
\def \ex {\mathrm{e}}
\def \ep {\varepsilon}
\def \h {\frac{1}{2}}
\def \z {\zeta}
\def \b {\underline{\beta}}
\def \W {W_{1+\infty}\ }
\def \A {{\mathcal A} }
%def \S {{\mathcal S}_{ds}\,}
\def \d {\delta}
\def \g {\underline{\gamma}}
\def \D {\Delta}
\def \o {\underline{\omega}}
\def \la {\langle}
\def \ra {\rangle}
\def \V {{\mathcal V}}
\def \w {{\mathcal W}}
\def \T {{\mathcal T}}
\def \t {\tau}
\def \th {\theta}
\def \B {{\mathcal B}}
\def \O {{\mathcal O}}
\def \H {{\mathcal H}}
\def \Lc {{\mathcal L}}
\def \Rset {{\mathbb R}}
\def \Cset {{\mathbb C}}
\def \Z {{\mathbb Z}}
\def \N {{\mathbb N}}
\def \G   {\Gamma}
\def \dd {\mathrm{d}}
\def \mod {\, \mathrm{mod}\, }
\def\U1{{\widehat{u(1)}}}
\def \PF {\mathrm{PF}}
\def \IP {\mathrm{IP} }
\def \ch {\mathrm{Ch} }

\def \PFC {${\mathcal{P}\mathcal{F}_k}$}


%\voffset -1cm
%\hoffset -0.5cm
%\topmargin 0.0cm
%\headheight 18pt
%\headsep 1.0cm
\textwidth 14cm
\textheight 20cm
%\oddsidemargin -0.5cm
%\evensidemargin -0.5cm
\def\thefootnote{\fnsymbol{footnote}}


\numberbysection
%\input proof.tex

\begin{document}

%- preprint front page --------------------------

\begin{titlepage}
\begin{center}
{\LARGE\bf Coset Construction of}

\vspace{.4cm}

{\LARGE\bf Parafermionic Hall States\footnote{
To appear in the proceedings of the 
International Seminar ``Supersymmetries and Quantum
Symmetries'', SQS'99, July 1999, Dubna, and of
the ``6th Wigner Symposium'', Istambul, August 1999.}
}

\vspace{.6cm}

A. Cappelli\ ,
\qquad L. S. Georgiev\footnote{
On leave of absence from Institute for Nuclear
Research and Nuclear Energy, Tsarigradsko Chaussee 72,
BG-1784 Sofia, Bulgaria}

\vspace{.1cm}

\normalsize\textit{ I.N.F.N. and Dipartimento di Fisica,}

\normalsize\textit{ Largo E. Fermi 2, I-50125 Firenze, Italy}

\vspace{.3cm}

I. T. Todorov

\vspace{.1cm}

\normalsize\textit{ Institute for Nuclear Research and Nuclear Energy,}
\\
\normalsize\textit{ Tsarigradsko Chaussee 72, BG-1784 Sofia, Bulgaria}
\end{center}

\vspace{.8cm}



\begin{abstract}
Fractional quantum Hall fluids with fillings $2 < \nu <3$
have been recently proposed which 
generalize the Pfaffian state ($\nu=2+1/2$) into a hierarchy of
states with parafermionic excitations. 
We describe here the corresponding  $\Z_k$-parafermion conformal 
field theory by means of the coset construction 
$\widehat{su(k)_1}\oplus\widehat{su(k)_1}/\widehat{su(k)_2}$.
This extends our earlier derivation of the Pfaffian state
from a ``parent'' state with abelian affine symmetry plus a 
projection of degrees of freedom.
The numerator of the coset
is actually a rational conformal field theory made out of
$(2k-2)$ scalar fields and a specific extended symmetry.
The coset construction projects out some neutral Hall edge excitations 
while preserving the filling fraction; it also respects the $\Z_k$ parity rule 
coupling neutral and charged excitations in the parent abelian theory.
\end{abstract}

\end{titlepage}

% -----------------------------------------------


\begin{center}
{\LARGE\bf Coset Construction of}

\vspace{.4cm}

{\LARGE\bf Parafermionic Hall States}

\vspace{.6cm}

A. Cappelli\footnote{E-mail: andrea.cappelli@fi.infn.it} \ ,
\qquad L. S. Georgiev\footnote{E-mail: georgiev@fi.infn.it.
On leave of absence from Institute for Nuclear
Research and Nuclear Energy, Tsarigradsko Chaussee 72,
BG-1784 Sofia, Bulgaria}

\vspace{.1cm}

\normalsize\textit{ I.N.F.N. and Dipartimento di Fisica,}

\normalsize\textit{ Largo E. Fermi 2, I-50125 Firenze, Italy}

\vspace{.3cm}

I. T. Todorov\footnote{E-mail: todorov@inrne.bas.bg}

\vspace{.1cm}

\normalsize\textit{ Institute for Nuclear Research and Nuclear Energy,}
\\
\normalsize\textit{ Tsarigradsko Chaussee 72, BG-1784 Sofia, Bulgaria}
\end{center}

\vspace{.3cm}



\begin{abstract}
Fractional quantum Hall fluids with fillings $2 < \nu <3$
have been recently proposed which 
generalize the Pfaffian state ($\nu=2+1/2$) into a hierarchy of
states with parafermionic excitations. 
We describe here the corresponding  $\Z_k$-parafermion conformal 
field theory by means of the coset construction 
$\widehat{su(k)_1}\oplus\widehat{su(k)_1}/\widehat{su(k)_2}$.
This extends our earlier derivation of the Pfaffian state
from a ``parent'' state with abelian affine symmetry plus a 
projection of degrees of freedom.
The numerator of the coset
is actually a rational conformal field theory made out of
$(2k-2)$ scalar fields and a specific extended symmetry.
The coset construction projects out some neutral Hall edge excitations 
while preserving the filling fraction; it also respects the $\Z_k$ parity rule 
coupling neutral and charged excitations in the parent abelian theory.
\end{abstract}

%-1------------------------------------------

\section{Introduction}

Conformal field theories have been successfully applied to
describe the universal properties of quantum Hall states \cite{prange},
such as symmetries, quantum numbers and low-energy dynamics of 
edge excitations  \cite{wen}.
The simplest Laughlin states with filling fraction $\nu=1,1/3,1/5,\dots$
are well understood in terms of  abelian conformal theories
with central charge $c=1$; most of our understanding
of edge excitations, including their fractional statistics
 and dynamics, has been drawn from these examples.
Furthermore, recent experiments have confirmed 
the predictions of the abelian conformal theories \cite{expe}.

The theoretical studies have now addressed more involved
Hall states, such as that occurring at $\nu=5/2$. 
This {\it plateau} in the second Landau level has no analogue in
the first level (i.e. at $\nu=1/2$); thus, it should be
caused by some new dynamical mechanism.
In Ref.\cite{pfaf}, it was proposed that the electrons form {\it pairs},
in a way similar to BCS pairs in superconductors;
these bosonic pairs can then form a
Laughlin fluid with even denominator filling fraction.
The ($p$-wave) pairing of (spin-polarized) electrons 
is represented in the ground-state wave function 
by the Pfaffian term ${\rm Pf}\left( 1/(z_i -z_j )\right)$,
which involves all possible pairs of electron coordinates.
(Other pairings were also proposed with different relative
angular momentum and spin polarization of electrons.)

There is a well-established relation between the Hall
states and conformal field theory, which says that
the analytic part of electron wave functions should correspond to
correlators of conformal fields.
Following Ref.\cite{pfaf}, the Pfaffian term can be reproduced by
the correlator of Majorana fermions in the $c=1/2$ conformal field theory,
i.e. the critical Ising model,
plus the usual $c=1$ boson theory accounting for the charge of excitations.

An interesting feature of the Pfaffian Hall state is that it possesses
excitations with {\it non-abelian} fractional statistics:
namely, the adiabatic transport of one excitation
around another is described by a multi-dimensional unitary
transformation acting on a multiplet of degenerate wave-functions,
rather than by the multiplication by a sign  (fermi statistics)
or a phase (abelian fractional statistics).
The non-abelian statistics is easily understood in the Ising conformal theory:
the spin field $\sigma$ possesses the operator-product expansion
$\sigma \cdot \sigma \sim {\rm Id} + \psi$, with two terms in the
right hand side; therefore, multi-spin correlators expand into several
terms (the conformal blocks), which transform among themselves
under monodromy.

Numerical analyses have shown that the Pfaffian
state has a rather good overlap with the exact ground state
at $\nu=5/2$ \cite{killhr}\footnote{
In contrast with the previous expectations
favoring the Haldane-Rezayi paired state.};
therefore, there is an exciting possibility that
new phenomena such as pairing and non-abelian statistics
could be experimentally observed 
at sufficiently low temperatures and $2 <\nu <3$.

Read and Rezayi have recently proposed \cite{rr} a generalization of
the Pfaffian to a hierarchy of states in the second
Landau level, which are described by the $\Z_k$-parafermion
conformal theories \cite{zf}. They occur at filling fractions,
\beq\label{0.1}
\nu \equiv 2 + \nu_k(M) = 2+ \frac{k}{kM+2}, \quad k=2,3,\ldots
\quad M=1,3,5,\dots \ .
\eeq
(The Pfaffian state corresponds to the value $k=2$ (and $M=1$) in this series.)
The same authors found that the parafermionic states have good 
overlap with the numerical exact ground states at $\nu=13/5, 8/3$, i.e.
for $M=1$ and $k=3,4$ \cite{rr}.
Furthermore, Hall plateaux at these filling values have been 
experimentally observed \cite{exper} by cooling the sample
at extremely low temperatures.
As shown in Ref.\cite{rr}, the $\Z_k$-parafermion Hall states 
describe an interesting dynamics: the
Hall fluid is made by clusters of $k$ electrons (e.g. pairs for $k=2$) 
and there are excitations with non-abelian statistics; other 
properties were discussed in Ref.\cite{gura}.

The problem we are addressing is to relate this (universal)
Hall dynamics to the conformal field theory data of $\Z_k$ parafermions.
For example, we would like to find conformal theory motivations
for the quantum numbers of these Hall states and for the
mechanism of clustering which actually yields the 
parafermions\footnote{
On the other hand, the non-abelian statistics can be
easily understood from the operator-product expansion
of parafermion fields.}.

Our general idea is to describe the non-abelian Hall states and
the associated conformal theories by introducing a {\it parent} abelian theory
(with the same filling fraction) and a projection leading to
the non-abelian theory.
The reason is that the Hall physics is well understood
in the abelian theories, such as the Laughlin fluids; 
furthermore, the projection may have a
physical interpretation, which could be useful to explain
pairing and non-abelian statistics.

We recall that an abelian theory is a rational
conformal field theory based on the $\U1^n$-current algebra,
spanned by $n$ abelian currents $J^i(z)$, extended by
vertex operators $Y(\l,z)=\np{\exp(i\phi(\l,z))}$ whose charges
$\l$ form an $n$-dimensional lattice $\G$.
This lattice, satisfying some physical conditions summarized in Section 2, 
has been called {\it chiral quantum Hall lattice} in Ref.\cite{fro}.
The non-equivalent quasi-holes in the Hall fluid correspond to the irreducible
representations of the extended algebra and are labeled by
the charges $\l\in\G^*/\G$ where $\G^*$ is the dual
lattice.
%\footnote{We recall that $\G^*/ \G$ is a finite abelian group
%of order $(\det\G)$ whose multiplication represents the fusion rules.}.
For example, the $\widehat{su(k)_1}$ theory is the extension
of $\U1^{k-1}$ theory with the $A_{k-1}$ root lattice. 
A complete classification of (low-dimension) chiral Hall lattices
has been obtained in Ref.\cite{fro}.

In Ref. \cite{cgt}, we have already shown that the Pfaffian state
can be described in terms of an abelian lattice theory 
by projecting out some neutral degrees of freedom.
A crucial step in this process is the decoupling of charged and
neutral sectors of the theory ((iso)spin-charge separation), which
cannot be done globally but give raise to a selection rule
(the {\it parity rule}). 
In addition, the projection should preserve the locality of
physical excitations, such as the electrons \cite{cgt}.

Here, we would like to briefly describe the generalization
of this approach to the Read-Rezayi states\footnote{
See Ref.\cite{cgt2} for a detailed account of this work.}.
We show that the $\Z_k$ parafermions can be
obtained by coset construction \cite{gko} from a specific 
abelian Hall state, whose lattice is {\it maximally symmetric}
in the sense of Ref.\cite{fro}, where it has been denoted by
$(M+2 \; | \; {}^{\L_1}A_{k-1} \ {}^{\L_1}A_{k-1})$.
The associated extended symmetry includes the
$\U1\oplus\widehat{su(k)_1}\oplus \widehat{su(k)_1}$ affine algebra.
The $\Z_k$-parafermion theory $PF_k$ is obtained as follows:
\beqa\label{1.4}
&&\PF_k=
\frac{\widehat{su(k)_1}\oplus \widehat{su(k)_1}}{\widehat{su(k)_2}},
\nn
&& c_{\PF_k}=c_{\widehat{su(k)_1}} + c_{\widehat{su(k)_1}} -
c_{\widehat{su(k)_2}} =\frac{2(k-1)}{(k+2)}  .
\eeqa
The two $su(k)$ symmetries of the parent state can be 
associated to {\it layer} and {\it iso-spin} quantum numbers \cite{cgt2}.
After the projection, there remain a $Z_k$ charge, 
which is the parafermion ``number'';
the $\Z_k$ parity rule relates this number (modulo $k$) to
the fractional charge of the quasi-holes of the resulting 
parafermion Hall fluid.
The coset construction allows us to construct all
super-selection sectors corresponding to the topologically non-equivalent
quasi-holes (the irreducible representations),
compute their characters  and
write the  partition functions for the Read--Rezayi states.

Other projective constructions of non-abelian Hall states can be found in
the Refs.\cite{schout}, including the different coset construction
$PF_k = \widehat{su(2)_k}/\U1$.


%-2---------------------------------------

\sectionnew{Parent abelian theory for the para\-fermion state}

%-2.1-------------------------------------

\subsection{Maximally symmetric $(2k-1)$-dimensional lattice}

In this Section, we describe the abelian theory
$\U1\oplus su(k)_1 \oplus su(k)_1$ with central charge
$c=2k-1$, which reproduces the same
filling fraction (\ref{0.1}) of the parafermion theory; 
the corresponding chiral Hall lattice $\G$ is an odd integral lattice
containing the electron charge vector $\q$ of square length
(twice the electron dimension):
\beq\label{0.2}
|\q|^2\equiv (\q|\q)=M+2 \; (=2\D_{\mathrm{el}}).
\eeq
In any chiral Hall lattice, one defines the {\it charge} vector $\Q$ of the 
dual lattice $\G^*$, which sets the {\it electric charge} 
of any lattice point and satisfies the following defining conditions: 
(i) it is {\it primitive} (i.e., not a multiple of any other vector
$\q^*\in\G^*_{2k-1}$); (ii) it is related to $\nu_k$ in (\ref{0.1})
and $\q$ by:
\beq\label{0.3}
|\Q|^2=\nu_k, \quad (\Q|\q)=1;
\eeq
(iii) it obeys the {\it charge--statistics relation} for fermion excitations:
\beq\label{0.3b}
(-1)^{(\Q|\q)}=(-1)^{|\q|^2} \qquad {\rm for \ any} \ \q\in\G.
\eeq

A convenient basis $\{\q,\a_i,\b_j\}$ for $\G$ is given by the electron
charge vector $\q$ (satisfying (\ref{0.2}) (\ref{0.3})) and the root vectors
$\a_1,\ldots ,\a_{k-1},\b_1,\ldots,\b_{k-1}$ of $su(k)\oplus su(k)$ with
standard inner products:
$(\a_i|\a_i)=(\b_i|\b_i)=2$, $(\a_i|\a_j)=(\b_i|\b_j)=-1$,
for $i\neq j=1,\dots ,k-1$.
The {\it symmetry algebra} $su(k)\oplus su(k)$ is assumed to be
{\it electrically neutral}, i.e. 
$(\Q|\a_i)=(\Q|\b_i)=0$, $i=1,\ldots,k-1$;
therefore, $\Q =(1,0 \dots 0)$ in the dual basis.
It then follows that the electron charge vector $ \q $
decomposes as follows:
\beq\label{1.3}
\q=\frac{1}{\nu_k}\Q +\o, \quad (\Q|\o)=0, \quad
 (\a_i|\o)=\d_{i1}=-(\b_i|\o), \quad |\o|^2=2\frac{k-1}{k};
\eeq
thus, $\o$ is an $su(k)\oplus su(k)$  weight.
(The sign convention in the relation $(\b_1|\q)=(\b_1|\o)=-1$
differs from the one in \cite{cgt} and is chosen to fit
the standard coset projection -- see Section 3.)

The Gram matrix in this basis takes the form:
\beq
G_\G =\left[
\begin{array}{c|c|c}
M+2 &  1 \; 0  \cdots  \ 0 & -1 \;  0  \cdots \ 0\cr \hline
\begin{array}{c}
1 \cr 0 \cr \vdots \cr 0
\end{array} &  C_{k-1} & 0 \cr \hline
\begin{array}{c}
- 1\cr 0 \cr \vdots \cr 0
\end{array} &  0 & C_{k-1}
\end{array} \right]\ ,
\eeq
with $C_{k-1}$ the Cartan matrix of the $A_{k-1}$ algebra.

In our previous analysis of the Pfaffian state ($k=2$) \cite{cgt},
we actually considered {\it two} parent abelian
states; the corresponding lattices were the maximally symmetric one
described above and the
the two-dimensional lattice $\tilde{\G}$ of the ($331$) paired
Hall state, with Gram matrix $G_{\tilde{\G}}={3 \ \ 1\choose 1\ \ 3}$.
The former lattice contains the latter and one can devise a two-step
projection connecting all the three theories:
this is indeed the coset construction (\ref{1.4}) for $k=2$ \cite{cgt2}.
In the general $k$ case, $\tilde{\G}$ extends to 
a $k$-dimensional lattice, which leads to another parent theory for the
parafermions; this is, however, completely independent of $\G$,
and its projection to the parafermions is not known. 

The unitary representations of the chiral algebra $\A(\G)$ 
describe the edge excitations of the Hall fluid;
they are labeled by the points of the dual lattice
$\G^*$. This is manifestly not decomposable into orthogonal
sublattices of charge and neutral excitation.
Nevertheless, this decomposition (physically meaning the isospin-charge
separation) can be achieved at the expenses of enlarging the lattice
and introducing a selection rule (the $\Z_k$ {\it parity rule}).
We introduce the decomposable sublattice 
$L\subset \G$ of index $k$ spanned by the vectors:
\beq\label{Lbasis}
\{ \e:= k(\q-\o)=(kM+2)\Q,\; \a_i,\;\b_j \}.
\eeq
It splits
into 3 mutually orthogonal sublattices:
\beq\label{1.5}
L=(kM+2)\Z\Q \oplus A_{k-1}\oplus A_{k-1} \ .
\eeq
We have
\beqa\label{1.6}
L\subset \G\subset \G^* \subset L^*, &\quad&
\G=\{ \g=\l+l\q \; ; \; \l\in L, \; 0\leq l \leq k-1 \}, \nn
&& L^*/\G^* \simeq \G/L \simeq \Z_k;
\eeqa
indeed, the determinants of the Gram matrices of $L$ and $\G$
(which give the number of sectors of the corresponding RCFT) are:
\beq\label{1.7}
|L|=(kM+2)^2 |\Q|^2 |C_{k-1}|^2=(kM+2)k^3 =k^2 |\G|.
\eeq
The isospin-charge separation of excitations is achieved
in the decomposable dual lattice $L^*$, whose physical points
(corresponding to those of $\G^*$) obey a selection rule $mod \ k$.
This factorized description of the excitations in the abelian theory
is crucial for allowing the projection of neutral degrees 
of freedom leading to the parafermion theory (see Section 3).

%-2.2---------------------------------

\subsection{Unitary representations of the chiral algebra $\A(\G)$ }

The irreducible unitary representations of the
chiral algebra $\A(L)$ can be expressed as
$\Z_k$-invariant products of fundamental representations of
$\widehat{su(k)_1}\oplus \widehat{su(k)_1}$  times chiral vertex
operators carrying charge $(n/k)\Q$ ($n\in\Z/(k(kM+2)\Z)$);
these representations are labeled by the elements of the abelian group
$L^*/L$. We shall choose a vector $\l\in L^*$ in each coset in $L^*/L$
of the form:
\beq\label{1.8}
\l=\frac{l}{k}\Q +\L_\mu^{(\alpha)} -\L_\nu^{(\beta)},
\quad \mu,\nu=0,1,\ldots, k-1, \quad 2|l|\leq k(kM+2).
\eeq
Here  $\L_\mu^{(\alpha)} $ ($\L_\nu^{(\beta)}$) are the fundamental
weights (including 0) of the first (respectively the second) $su(k)$
factor. The conformal dimension of the representation $\l$ is:
\beq\label{1.9}
\D(\l)=\h |\l|^2 = \frac{l^2}{2k(kM+2)} + \frac{\mu(k-\mu)+\nu(k-\nu)}{2k}.
\eeq
\begin{prop}
A vector $\l$ of $L^*$ belongs to the sublattice $\G^*$ iff
$(\l|\q)\in \Z$. For $\l$ given by (\ref{1.8}), this is equivalent
to the relation:
\beq\label{1.10}
l+k(\o|\L_\mu^{(\alpha)}-\L_\nu^{(\beta)}) \in k\Z \quad \Leftrightarrow
\quad l-\mu-\nu=0 \ \ \mod k.
\eeq
\end{prop}
{\bf Proof.}
Since $\q\in \G$, if $\l\in\G^*$ the inner product $(\l|\q)$ should be
integer. For $\l\in L^*$ the converse is also true in view of (\ref{1.3}) and
(\ref{1.5}). Eq. (\ref{1.10}) is then a consequence of (\ref{0.3})
(\ref{1.3}) and (\ref{1.5}) and of the definition of fundamental weights
as a dual basis for $ \{ \a_i \} $ and $ \{ \b_i \} $:
$(\a_i|\L_j^{(\alpha)})=\d_{ij} =(\b_i|\L_j^{(\beta)})$ ,
$ i,j=1,\ldots,k-1$.
As a result, the inner products $(\L_i|\L_j)$ (for $\L$ belonging
to the weight space of the same $su(k)$ factor, $\alpha$ or $\beta$)
are expressed in terms of the inverse $A_{k-1}$ Cartan matrix.

Eq. (\ref{1.10}) provides the explicit form of the $\Z_k$ parity rule which
tells us when an excitation in the bigger lattice $L^*$ actually
belongs to the sublattice $\G^*$ of physical excitations of the 
Hall fluid.

We shall label the irreducible representations
 of $\A(\G )$  ( i.e., the elements
of $\G^* /\G$ )  by a pair $(m,\nu)$ where $m$ measures the minimal
charge of each irreducible representation 
so that $2|m| \leq (kM + 2) $, while $\nu \mod k$
characterizes the neutral part. To each pair $(m,\nu)$
there corresponds a set of $k$ weight vectors of type (\ref{1.8}), which
satisfy (\ref{1.10}):
\beqa\label{2.13}
(m,\nu) \leftrightarrow \l_l(m,\nu) &=& \frac{m + l(kM+ 2)}{k}\Q
+ \La_{m + l - \nu} - \Lb_{l+\nu} \  ,\nn
&& (\L_\mu \equiv \L_{\mu \mod  k},  l \mod k).
 \eeqa
The length squares of these vectors differ by integers \cite{cgt2}.

Using the representation of each vector $\g\in \G$ as a sum
of a $\l\in L$ and a {\it gluing vector} \cite{fro} $l \q\in \G$
(see (\ref{1.6})) we can write the representation space
$\H^\G_{\g^*}$ of $\A(\G)$ ($\g^*\in\G^* /\G$)
as direct sums of representation spaces of $\A(L)$,
\beq\label{spaces}
 \H^\G_{\g^*} = \bigoplus_{l=0}^{k-1} \H^L_{\g^*+l\q}.
\eeq
This form of $\H^\G_{\g^*}$ allows to write its characters
as a sum of characters of $\A(L)$ modules.
Denote by $\chi_\nu(\t)$ the (restricted)  $\widehat{su(k)_1}$ characters
\cite{kt},
\beq\label{1.16}
\chi_\nu(\t,A_{k-1})=\frac{1}{(\eta(\t))^{k-1}}
\sum_{\g\in A_{k-1}} q^{\h|\L_\nu +\g|^2}, \quad q=\ex^{2\pi i \t},
\quad \nu=0,1,\ldots, k-1,
\eeq
($\eta$ is the Dedekind $\eta$-function). Taking the standard
notation (also used in \cite{cgt}) for the 1-dimensional lattice characters,
\beq\label{1.17}
K_l(\t,\zeta;m)=\frac{1}{\eta(\t)} \sum_{n\in\Z}
q^{\frac{m}{2}(n+\frac{l}{m})^2} \ex^{2\pi i \zeta (n+\frac{l}{m})},
\eeq
we can write the characters $\chi_{\l}(\t,\zeta)$
of the representation $\l$ (\ref{1.8}) in terms of the labels
(\ref{2.13})
as a sum of factorized $L$-characters,
\beq\label{1.18}
\chi_{m \nu}^\G(\t,\zeta)= 
\sum_{s=0}^{k-1} K_{m +s(kM+2)}\left(\t,k\zeta;k(kM+2)\right)
\chi_{m+s-\nu}^{(\alpha)}(\t)   \chi_{s+\nu}^{(\beta)}(\t),
\eeq
with $m$ mod $(kM+2)$ and $\nu$ $ \mod k $.
The resulting set of $k(kM+2)$ functions is covariant under (weak)
modular transformations generated by
$T^2: (\t,\zeta)\rightarrow (\t+2,\zeta)$ and
$S:(\t,\z) \rightarrow (-1/\t,\zeta/\t)$, which are the modular
properties suitable for quantum Hall systems \cite{cz}.


%-3-------------------------------------------

\sectionnew{The $\Z_k$-parafermion coset and its representations}

In this Section, we describe the coset construction (\ref{1.4});
this can be done in each of the $k$ sectors (\ref{spaces}) of the abelian
theory according to the $\Z_k$ parity rule.

%-3.1---------------------------------------

\subsection{$\Z_k$ selection rule for triples of su(k) weights.
Conformal dimensions of coset representations}

The $\PF_k$ coset module (\ref{1.4}) is labeled, in principle, by a triple
$(\L_{\alpha},\L_{\beta};\L)$; the pair $(\L_{\alpha},\L_{\beta})$ of
fundamental su(k) weights and the level 2 weight $\L$ fix an irreducible
unitary representation
of the numerator and of the denominator current algebra, respectively.
The tensor product of irreducible $\widehat{su(k)_1}$-modules corresponding
to the numerator in the right hand side of Eq. (\ref{1.4}) splits into a
direct sum of $\PF_k$ and $\widehat{su(k)_2}$-modules:
\beq\label{3.1}
\H^{(1)}_{\L_{\alpha}} \otimes  \H^{(1)}_{\L_{\beta}} = \bigoplus\limits_{\L}
\H(\L_{\alpha},\L_{\beta};\L)\otimes \H^{(2)}_{\L} \qquad
\left( \alpha,\beta=0,\ldots,k-1, \; \L_0=0\right).
\eeq
Not all triples $(\L_{\alpha},\L_{\beta};\L)$  are admissible (i.e.,
correspond to non-empty coset modules) and different
{\it admissible triples} may refer to {\it equivalent representations}.
The following statement is a
specialization of results on field identification (based on the use of simple
currents) obtained in Ref.\cite{schw}.
\begin{prop}
Admissible triples are characterized by the conservation of the $\Z_k$
charge (the $k$-ality):
\beq\label{3.2}
[\L]=\sum_{i=1}^{k-1} i \ \lambda_i \quad
\mathrm{for} \quad \L=  \sum_{i=1}^{k-1}  \lambda_i \L_i ;
\eeq
more precisely, the triple $(\L_{\alpha},\L_{\beta};\L)$  is
admissible iff:
\beq\label{3.3}
 [\L_{\alpha}]+[\L_{\beta}]=[\L] \ \ \mod k , \qquad
\mathrm{i.e.,} \quad \alpha +\beta =[\L] \ \ \mod k.
\eeq
There are thus $k {k+1 \choose 2}$ admissible triples of the form
$(\L_{\alpha},\L_{\beta};\L_{\alpha +\kappa}+\L_{\beta -\kappa})$
where all indices are taken $\mod k$. They split into $ {k+1 \choose 2}$
families of equivalent triples of the form:
\beq\label{3.4}
(\L_{\alpha+\s},\L_{\beta+\s};\L_{\alpha +\kappa+\s}+\L_{\beta -\kappa+\s}) ,
\quad \s=0,\ldots, k-1.
\eeq
As a result, the number $N(\PF_k)$ of parafermionic coset sectors coincides
with the number of unitary irreducible representation  $N(\widehat{su(k)_2})$ of the level 2 current algebra:
$ N(\PF_k)=N(\widehat{su(k)_2})=   {k+1 \choose 2}$.
\end{prop}
We can define a representative for each family in Eq.(\ref{3.4})
by choosing a value for $\s$: we set
$\beta+\s=0 \mod k$, thus normalizing the
second fundamental weight to zero. 
Therefore, the equivalent classes of triples are labeled by
the level 2 weight, say $\L_\mu+\L_\nu$, $\mu\leq \nu$; we have:
\beq\label{3.6}
(\L_{\mu+\nu \mod k}, 0;\L_\mu+\L_\nu ) \quad \Longleftrightarrow
\L_\mu+\L_\nu \quad (\mu\leq \nu).
\eeq
Ultimately, these labels characterize the parafermion representations.
We end up with the following characterization of $\PF_k$ coset modules which
appears to be new \cite{cgt2}.
\begin{prop}
The parafermionic coset modules are in one to one correspondence with
sums $\left(\L_\mu+\L_\nu \right)$ of $su(k)$ fundamental weights
($0\leq \mu \leq \nu \leq k-1$). Their conformal weights are given by:
\beqa\label{3.7}
\D^{(k)}_{\mu\nu}&=&\h \left| \L_\mu +\L_\nu \right|^2 -
\D_2\left( \L_\mu +\L_\nu \right) \nn
& = &
\frac{\mu(k-\nu)}{k}+\frac{(\nu-\mu)(k+\mu-\nu)}{2k(k+2)}
\quad \mathrm{for} \quad \left(0\leq\mu\leq\nu<k \right),
\eeqa
where $\D_2(\L)$ is the dimension of the representation of the level 2
weight $\L$ of $\widehat{su(k)_2}$. The decomposition of a product of level-$1$
characters corresponding to the tensor product expansion (\ref{3.1}) has the
form:
\beq\label{3.8}
\chi^{(1)}(\L_{\alpha})   \chi^{(1)}(\L_{\beta}) =
\sum_{\gamma} \ch\left(\L_{\alpha-\beta +\gamma} + \L_{k-\gamma}\right)
\chi^{(2)} \left(\L_{\alpha-\beta +\gamma} + \L_{k-\gamma}\right) \ ,
\eeq
where $\ch(\L)$ is the coset character and $\gamma$ takes
$\IP(k/2)+1$ values corresponding to distinct sums
of pairs of weights ($\IP(x)$ stands for the integer part of the real number
$x$).
\end{prop}
{\bf Proof.}
Eq. (\ref{3.7}) can be verified by using the identity
$ \D_1(\L_\mu) + \D_1(\L_{k-\nu}) -   \D_2(\L_{\mu+k-\nu})  =
\D^{(k)}_{\mu\nu} $
and observing that the triple $(\L_\mu,\L_{k-\nu};\L_{\mu+k-\nu})$
is equivalent to  $(\L_{\mu+\nu},0;\L_{\mu}+\L_{\nu})$.
The triples appearing in Eq. (\ref{3.8}) (as arguments of the pair of
$\chi^{(1)}$ and $\chi^{(2)}$) are, clearly, admissible. It is
straightforward to verify that the difference of conformal  weights
of the two sides is an integer.
The number of terms in the expansion (\ref{3.8}) is independent
of $\alpha$  and $\beta$. For the vacuum representation we have:
\[
\chi^{(1)}(\L_0)   \chi^{(1)}(\L_0) =  \ch(2\L_0) \chi^{(2)}(2\L_0)+
\sum_{\gamma=1}^{\IP(\frac{k}{2})}  \ch\left(\L_{\gamma} + \L_{k-\gamma}\right)
\chi^{(2)} \left(\L_{\gamma} + \L_{k-\gamma}\right) \ ,
\]
where $\D^{(k)}_{\gamma, k-\gamma}+ \D_2(\L_\gamma +\L_{k-\gamma})=\gamma$
for $\gamma\leq k-\gamma$. \eop

%-3.2-----------------------------------------

\subsection{Parafermionic Hall fluids. The parafermion $\Z_k$ charge}


The chiral algebra $\A_k$ of the $\Z_k$-parafermion Hall fluids, 
which reproduces the filling factor (\ref{0.1}), is determined from:
\beq\label{3.11}
\A_k \otimes \A(\widehat{su(k)_2})=\A(\G).
\eeq
This is obtained by a standard coset projection from the lattice
theory of Section 2. In particular, the lattice
characters (\ref{1.18}) are projected into:
\beq\label{3.12}
   \chi_{m \nu } (\tau, \z) = \sum_{s \mod  k} K_{m+ s(kM+2)}
(\tau , k\z; k(kM+2)) \ch(\tau, \L_{s+m-\nu} +\L_{s+\nu} ) .
\eeq

The coset projection only preserves a single $\Z_k$ symmetry of 
the original product $\Z_k \times \Z_k$ of the
centres of the two $SU(k)$ groups; it is the difference:
\beq\label{3.13}
[\L_{\alpha +\s}]- [\L_{\beta +\s}]= (\alpha - \beta) \quad \mod k \ , 
\eeq
which defines the $\Z_k$ charge (or ``number'') of parafermions.

Moreover, the $\Z_k$ parity rule of the parafermion Hall fluids
is inherited from the parent abelian fluids, namely Eq.(\ref{1.8}): 
it states that the physical Hall excitations possess
parafermion number (\ref{3.13}) 
equal ($mod \ k$) to the number of ``fractional units'' of electric charge 
$l \in \Z_{kM+2}$ in Eq.(\ref{1.8}):
\beq\label{p-rule}
\alpha -\beta =l \quad {\rm mod}\ k\ .
\eeq

The coset representation $2\L_\nu$, corresponding to $\mu=\nu$,
are the {\it ``parafermionic currents"} of Fateev and Zamolodchikov
\cite{zf}. They all have quantum dimension 1 and obey $\Z_k$ fusion rules:
\beq\label{3.14}
2\L_{\mu} \times 2\L_{\nu}  \sim   2\L_{(\mu+\nu) \mod k} .
\eeq
The (non-local) parafermionic currents give rise
to an ``anyonic" chiral algebra, say, \PFC
whose bosonic (integer dimension fields') subalgebra can be
 identified with the coset chiral algebra $\PF_k$ (\ref{1.4}). 
The parafermionic algebra \PFC admits $k$ unitary irreducible representations, 
labeled by an integer  $\r \mod  k$, with conformal weights,
\beq\label{3.15}
\D_\r = \frac{\r(k-\r)}{2k(k+2)}, \quad \r= 0,1,...,k-1.
\eeq
Each of these splits into $(k-\r)$ unitary irreducible representation 
of the bosonic subalgebra
$\PF_k$ whose conformal weights exceed (\ref{3.15}) by an integer multiple
of $1/k$.  Comparing (\ref{3.15}) with (\ref{3.7})
we see that $\r$ can be identified  with $(\nu-\mu)$. For each $\r$
in the range  (\ref{3.15}) there are  exactly $(k-\r)$ pairs $(\mu,\nu)$
satisfying $0\leq \mu\leq\nu\leq k$, $\r=\nu-\mu$; they generate all
different conformal weights (\ref{3.7}). According to \cite{schil} the
characters of the resulting coset modules are given by
$\ch(\tau,\L_\mu+\L_\nu)= \ch_{\r l}(\tau)$, where:
\beq\label{3.16}
\ch_{\r l}(\tau)=q^{\D_\r-\frac{c_k-1}{24}}
\sum\limits_{\underline{n}\in \N_l}
\frac{q^{\underline{n}\cdot C^{-1}\cdot (\underline{n}- \L_{\r} ) }}
{ (q)_{n_1}\cdots (q)_{n_{k-1}}},
\quad l\geq \r.
\eeq
In this equation, $\D_\r$ is the  $\PF_k$ weight (\ref{3.15}), $(c_k -1)$
is the parafermion central charge (cf. (\ref{1.4})),
$ (q)_n=\prod_{j=1}^n (1-q^j)$,
$\N_l=\left\{ \underline{n}=(n_1,\ldots,n_{k-1})\right.$; 
$n_i \in \Z_+ ;$
$ \left. n_1+2n_2+\cdots + (k-1)n_{k-1}=l \mod k\right\}$, and
$C^{-1}$ is the inverse of the $su(k)$ Cartan matrix.

The expression (\ref{3.16}) corresponds to a $\PF_k$ irreducible
component of the representation $\r$ of the non-local parafermionic
algebra \PFC. For fixed $\r$,
the values of $l$ yielding inequivalent $\PF_k$ modules are:
\beq\label{3.21}
l = \r, \r +1,\ldots,k-1 \; (\mathrm{for} \; \r =0,1, \ldots, k-1).
\eeq
The pair $(\r,l)$ is related to $(\mu,\nu)$ of (\ref{3.7}) by:
\beq\label{3.22}
\r = \nu - \mu , \quad l=\nu  \; \mathrm{or} \; \mu = l-\r, \nu =l
\Rightarrow 2l-\r \equiv m \mod  k.
\eeq
The {\it topological order} $N_k$  of 
the $\Z_k$-parafermion Hall fluid
is equal \cite{cz} to the number of independent characters (\ref{3.12})
of the corresponding conformal theory; 
it is given by the product of the range of m, i.e. ($kM + 2$), 
times the number ${ k+1 \choose 2}$ of $\Z_k$-parafermion
sectors and divided by $k$, due to the parity rule (\ref{p-rule}):
\beq\label{3.23}
N_k = \frac{1}{k}(kM + 2) { k +1 \choose 2} = \frac{k+1}{2}(kM + 2 ) \ .
\eeq
We thus recover the value obtained in Ref.\cite{rr} by other means.


%-4---------------------------------------

\sectionnew{The $k=3$ case}

Here we spell out our construction in the case of $k=3$
(the analysis of the Pfaffian state ($k=2)$ can be found in the 
Refs.\cite{cgt} \cite{cgt2}).
Consider the lattice model of Section 2 for
$k=3$, $M=1$: the  coset (\ref{1.4}) coincides in this case with
the $\Z_3$ Potts model (with $c_\PF=4/5$). We shall label its 6 sectors
by an integer $\lambda$,  $-2 \leq \lambda \leq 3$,  related to the pair
$(\mu,\nu)$  ($0\leq \mu,\nu \leq 2$) of Eq. (\ref{3.7}) as follows:
$1=(1,0)=(0,1)$, $ \; -1=(2,0)=(0,2)$, $\; 2=(1,1)$, $\; -2=(2,2)$,
$\; (\pm) 3=(1,2)=(2,1)$. 
The conformal dimensions $\D_\lambda$ are given by:
\beq\label{3.5'}
\D_0=0, \quad \D_{\pm 1}=\frac{1}{15}, \quad \D_{\pm 2}=\frac{2}{3},
\quad  \D_{\pm 3}=\frac{2}{5}.
\eeq
The 6 coset characters (\ref{3.16}) for $k=3$ can be written as,
\beqa\label{4.9}
\!\!\!\!\! ch_{2l}(\t)&=&\ch_{0,l}(\t) = q^{-\frac{1}{30}}
\sum_{n_1+2n_2=l \mod 3,\ \ (n_i\geq 0)}
\frac{q^{\frac{2}{3}\left(n_1^2+n_2n_2+n_2^2 \right)}}{(q)_{n_1}(q)_{n_2}},
\quad l=0,\pm 1 (\mod 3) \ ,\nn
\!\!\!\!\! ch_{2l-1}(\t)&=& \ch_{l+1,l+1}(\t) = 
q^{\frac{1}{30}} \sum_{n_1+2n_2=0 \mod 3}
\frac{q^{\frac{2}{3}\left(n_1^2+n_2n_2+n_2^2 \right) - 
\frac{2n_{l+1}+n_{2-l}}{3}}}{(q)_{n_1}(q)_{n_2}}, \quad l=0, 1\ , \nn
\!\!\!\!\! ch_{3}(\t)&=&\ch_{1,2}(\t) = 
q^{\frac{1}{30}} \sum_{n_1+2n_2=2 \mod 3}
\frac{q^{\frac{1}{3}\left[2n_1(n_1-1)+2 n_1 n_2 + n_2(2n_2-1) \right]}}
{(q)_{n_1}(q)_{n_2}}.
\eeqa
Their modular transformations will be omitted here and can be found
in Ref.\cite{cgt2}.
 
The $\Z_3$ parafermion Hall fluid with chiral algebra $\A_3$ defined by
(\ref{3.11}), filling factor $3/5$  and central charge $c=9/5$
has topological order 10 (according to (\ref{3.23}) ) and characters:
\beqa\label{3.8'}
\chi_{m \mu}(\t,\z)&=&\sum_{l=-1}^1 K_{m+5l} (\t,3\z;15) ch_{m+3\mu+2l}(\t)
\qquad (ch_\lambda(\t)=ch_{\lambda+6}(\t) ), \nn
&& m=0,\pm 1, \pm 2, \quad \mu=0,1.
\eeqa
Note that the charge ($K$) and neutral ($ch$) parts on the Hall fluid
characters in Eq.(\ref{3.8'}) combine according to the 
$\Z_3$ parity rule (\ref{p-rule}).
 
The minimal charges $Q_{m \mu}$ and minimal conformal
dimensions $\D_{m \mu}$ in the sectors (\ref{3.8'}) are:
\beqa\label{4.13}
 Q_{m \mu}=\frac{m}{5}, \quad &&\D_{m 0} =\frac{1}{10}{|m|+1 \choose 2},
%\quad \left( \D_{20}= \frac{3}{10}=3\D_{10} \right) 
\nn
&& \D_{01}=\frac{2}{5}=2\D_{2 1}, \quad \D_{11}=\frac{7}{10}.
\eeqa

The full characters (\ref{3.8'}) of the parafermion Hall fluid
transform linearly under the $S$ modular inversion $\tau \to -1/\tau$;
the associated matrix is:
\beq\label{3.10'}
S_{m\mu,m'\mu'}=\frac{\sqrt{3-\d}}{5}
(-1)^{\mu\mu' +m\mu'+m'\mu} \ex^{i\frac{\pi}{5} m m'}
\d^{P_{\mu+m+\mu'+m'}} \ ,
\eeq
with $m,m'=0,\pm1,\pm 2$,  $\mu,\mu'=0,1$,
$\delta=2 \cos (\pi/5)$ the golden ratio and
$P_{\lambda} = \left( 1-(-1)^{\lambda}\right)/2$.
In particular, the quantum dimension $d_{q}$ of the representation
$(m,\mu)$ is:
\beq\label{3.11'}
d_q(m,\mu)=
\left\{ \begin{array}{rll} 1 & \mathrm{for} \; \mu+m & \mathrm{even} , \\
\d & \mathrm{for} \; \mu+m & \mathrm{odd}\ . \end{array} \right.
\eeq
The presence of a non-integer (in fact, irrational) quantum dimension
$\d$ signals the non-abelian statistics of the quasi-particle
excitations in this Hall fluid. Actually, some fusion
rules yields more than one field in the r.h.s., for example:
\beqa\label{3.12'}
&&(1,0) \times (1,0) = (2,0) + (2,1)  \nn
&&(0,1) \times (0,1) = (0,0) + (0,1).
\eeqa
Further properties of the coset construction of 
$\Z_k$-parafermion Hall fluids, such as the ${\cal W}_k$ symmetry
\cite{ctz}, can be found in Ref.\cite{cgt2}.

\bigskip
{\bf Acknowledgments}

L. G. thanks Christoph Schweigert for fruitful discussions.
L. G. thanks I.N.F.N. and Dipartimento di Fisica, Firenze for hospitality;
I. T. thanks the Erwin Schroedinger Institute (ESI), Vienna and l'Institut des
Hautes Etudes Scientifiques (IHES), Bures-sur-Yvette, for hospitality
during the completion of this paper.
I.T. thanks the organizers of the
Seminar on "Symmetries and Quantum Symmetries", Dubna, July 1999 and of
the "6th Wigner Symposium", Istambul, August 1999 for inviting him to
present this talk.
L.G. and I.T. acknowledge partial support from the Bulgarian
National Foundation for Scientific Research under contract F-827.
A.C. acknowledges the partial support from the European
Community network programme FMRX-CT96-0012.
%%%%%%%%%%%%%%%%%%%%%%%%%%%%%%%%%%%%%%%%%%%%%
\def\NP{{\it Nucl. Phys.\ }}
\def\PRL{{\it Phys. Rev. Lett.\ }}
\def\PL{{\it Phys. Lett.\ }}
\def\PR{{\it Phys. Rev.\ }}
\def\CMP{{\it Comm. Math. Phys.\ }}
\def\IJMP{{\it Int. J. Mod. Phys.\ }}
\def\JSP{{\it J. Stat. Phys.\ }}
\def\JP{{\it J. Phys.\ }}
%%%%%%%%%%%%%%%%%%%%%%%%%%%%%%%%%%%%%%%%%%%%%%
\begin{thebibliography}{999}
%%%%%%%%%%%%%%%%%%%%%%%%%%%%%%%%%%%%%%%%%%%%%%
\bibitem{prange} For a review see: R. A. Prange and S. M. Girvin, 
                {\it The Quantum Hall Effect}, Springer Verlag, New York 
                (1990); S. Das Sarma and A. Pinczuk, {\it Perspectives 
                in Quantum Hall effects}, John Wiley, (1997).
\bibitem{wen}   For a review, see: X. G. Wen, \IJMP {\bf 6 B} (1992) 1711; 
                {\it Adv. in Phys.} {\bf 44} (1995) 405.
\bibitem{expe}  L. Saminadayar, D. C. Glattli, Y. Jin and
                B. Etienne, \PRL {\bf 79} (1997) 2526;
                R. de Picciotto, M. Reznikov, M. Heiblum, V. Umansky,
                G. Bunin and D. Mahalu, {\it Nature} {\bf 389} (1997) 162.
\bibitem{pfaf}  G. Moore and N. Read, \NP {\bf B 360} (1991) 362;
                see also B. Halperin, {\it Helv. Phys. Acta} {\bf 56}
                (1983) 1031.
\bibitem{killhr} R. H. Morf, \PRL {\bf 80} (1998) 1505; E. H. Rezayi and 
                F. D. M. Haldane, cond-mat/9906137.
\bibitem{rr}   N. Read, E. Rezayi, \PR {\bf B 59} (1998) 8084.
\bibitem{zf}  A.B. Zamolodchikov and V.A. Fateev,  Sov. Phys. JETP 
              {\bf 62} (1985) 215, Nucl. Phys. {\bf B 280}(1987) 644;
              D. Gepner, Z. Qui, \NP {\bf B285} (1987) 423.
\bibitem{exper} W. Pan et al., \PRL {\bf 83} (1999) 3530.
\bibitem{gura} V. Gurarie and E. Rezayi, cond-mat/9812288.
\bibitem{fro}  J. Fr\"{o}hlich and E. Thiran, \JSP {\bf 76} (1994) 209;
               J. Fr\"{o}hlich, U. M. Studer and E. Thiran, \JSP
               {\bf 86} (1997) 821;J. Fr\"{o}hlich, T. Kerler,
                U. M. Studer and E. Thiran, \NP {\bf B 453} (1995) 670.
\bibitem{cgt} A. Cappelli, L. S. Georgiev and I. T. Todorov,
               Commun. Math. Phys. {\bf 205} (1999) 657--689.
\bibitem{gko} P. Goddard, A. Kent,  D. Olive, Commun. Math. Phys.
              {\bf 103} (1986) 105.
\bibitem{cgt2} A. Cappelli, L. S. Georgiev and I. T. Todorov,
               in preparation.
\bibitem{schout} E. Fradkin, C. Nayak and K. Schoutens, cond-mat/9811005;
               X. G. Wen, cond-mat/9811111. 
\bibitem{kt} V.G. Kac and I.T. Todorov, Commun. Math Phys. {\bf 190} 
              (1997) 57.
\bibitem{cz}  A. Cappelli and G.R. Zemba, \NP {\bf B490} (1997) 595.
\bibitem{schw} J. Fuchs, A.N. Schellekens, C. Schweigert,
              \NP {\bf B461} (1996) 39.
\bibitem{schil} A. Schilling, {Nucl. Phys.} {\bf B459}, 393 (1996),
              {Nucl. Phys.} {\bf B467}, 247 (1996), and references therein.
\bibitem{ctz}  A. Cappelli, C. A. Trugenberger and G. R. Zemba,
               \NP {\bf B 396} (1993) 465, \NP {\bf B 448} (1995) 470; 
               Cappelli and G. R. Zemba, \NP {\bf B 540} (1999) 610;
               S. Iso, D. Karabali and B. Sakita, \NP {\bf B 388} (1992) 700.
\end{thebibliography}

\end{document}